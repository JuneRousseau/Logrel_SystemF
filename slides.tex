\documentclass[xcolor={usenames,dvipsnames}]{beamer}
\title{Formalizing logical relation for System F type safety in Coq}
\author{Bastien ROUSSEAU}
\date{2023}

\usepackage[utf8]{inputenc}
\usepackage{amsmath}
\usepackage{amssymb}
\usepackage{stmaryrd}
\usepackage{mathpartir}
\usepackage{pftools}

\usepackage{generalMacros}
\usepackage{syntaxColor}
\newcommand{\link}[1]{\href{#1}{\cstep}}

\newcommand{\lrp}[2]{\llbracket #2 \rrbracket_{#1}}
\newcommand{\lr}[3]{\llbracket #2 \rrbracket_{#1}(#3)}
\newcommand{\lrv}[2]{\lr{#1}{#2}{v}}
\newcommand{\typed}[3]{#1 \vdash #2 : #3}
\newcommand{\hstep}{\rightsquigarrow}
\newcommand{\step}{\rightarrow}
\newcommand{\mstep}{\step^{\ast}}

\let\originalleft\left
\let\originalright\right
\renewcommand{\left}{\mathopen{}\mathclose\bgroup\originalleft}
\renewcommand{\right}{\aftergroup\egroup\originalright}


\newcommand{\fst}[1]{\com{fst}~#1}
\newcommand{\snd}[1]{\com{snd}~#1}
\renewcommand{\tt}{\com{tt}}
\newcommand{\lam}[2]{\lambda #1.~#2}
\newcommand{\app}[2]{#1~#2}
\newcommand{\tlam}[1]{\Lambda#1}
\newcommand{\tapp}[1]{#1~\_}
\newcommand{\pair}[2]{\left(#1,\,#2 \right)}
\renewcommand{\ctx}{\ensuremath{\mr{K}}}
\newcommand{\empctx}{\bullet}
\newcommand{\subst}[3]{#1.\left[#2/#3\right]}


\newcommand{\tyunit}{\mr{unit}}
\newcommand{\ty}{\tau}
\newcommand{\typair}[2]{#1 \times #2}
\newcommand{\tyarrow}[2]{#1 \rightarrow #2}
\newcommand{\tyforall}[2]{\forall #1.~#2}

\newcommand{\exprctx}{\Gamma}
\newcommand{\tyctx}{\Delta}

\newcommand{\safe}{\ensuremath{\mr{Safe}}}
\newcommand{\norm}{\ensuremath{\mr{Norm}}}
\newcommand{\val}{\ensuremath{\mr{Val}}}

\newcommand{\sctx}{\xi}
\newcommand{\sfun}{\gamma}


\usepackage{appendixnumberbeamer}
% Theme
\usecolortheme{beaver}
\beamertemplatenavigationsymbolsempty
\setbeamercolor{palette primary}{bg=white,fg=black}
\setbeamercolor{palette secondary}{bg=white,fg=black}
\setbeamercolor{palette tertiary}{bg=white,fg=black}
\setbeamercolor{palette quaternary}{bg=white,fg=black}
\setbeamercolor{structure}{bg=white,fg=black}
\setbeamercolor{titlelike}{bg=white,fg=black}
\setbeamercolor{title}{bg=white,fg=black}
\setbeamercolor{title in head/foot}{bg=white,fg=black}
\setbeamercolor{section in head/foot}{bg=white,fg=black}
\setbeamercolor{subsection in head/foot}{bg=white,fg=black}
\setbeamercolor{frametitle}{bg=white,fg=black}
\setbeamertemplate{footline}[frame number]
\newcommand\xxsectiontitle[1]{\begin{center}\Huge{#1}\end{center}}
\renewcommand\section[1]{\begin{frame}[noframenumbering]{}\xxsectiontitle{#1}\end{frame}}
\setbeamertemplate{blocks}[rounded][shadow=true]
\setbeamercolor{block title}{bg=black!5, fg=black}

\begin{document}

\frame{\titlepage}

\begin{frame}
  \frametitle{Introduction}

% In the lecture, paper proof of type safety for System F using logical relation.
% In general, paper proof is prone to mistakes, so the project was to implement
% the logical relation in Coq.
% System F with unit and products.

  \begin{block}{Paper proof}
    \begin{itemize}
      \item Type safety System F (with unit and products)
      \item Syntactic method VS logical relations
      \item Paper proof: prone to mistakes
    \end{itemize}
  \end{block}

  \begin{block}{Implementation}
    \begin{itemize}
      \item Stronger guarantees
      \item Representations issues
      \item Improvement and extensions
    \end{itemize}
  \end{block}

\end{frame}

% Table of content
\begin{frame}{Outline}
  \begin{enumerate}
    \item Reasoning of System F
    \item Implementation
          \begin{itemize}
            \item Nameless binders
            \item Substitution lemma
            \item Different approach
          \end{itemize}
    % \item Extensions
  \end{enumerate}
\end{frame}

% In this section, we highlight the main steps of the proof.
% TODO i'm not really satisfied by this first section, I am not really
% sure about what I want to put inside it
% \section{Type safety on paper}%
\section{Reasoning on System F}%

\begin{frame}{System F}
  \begin{block}{Polymorphism}
    \begin{itemize}
      \item very expressive feature (compared to STLC)
      \item main motivation to use logical relation
    \end{itemize}
  \end{block}

  \begin{block}{Typing judgement (excerpt)}
    $\fbox{\typed{\tyctx; \exprctx}{e}{\ty}}$
    \begin{mathpar}
      \inferH
      {T-TAbs}
      {\typed{\alpha, \tyctx; \exprctx }{e}{\ty}
        \\
        \alpha \mathrm{\ is\ not\ free\ in\ }\exprctx}
      {\typed{\tyctx; \exprctx}{\tlam{e}}{\tyforall{\alpha}{\ty}}}
      \and

      \inferH
      {T-TApp}
      {\typed{\tyctx; \exprctx}{e}{\tyforall{\alpha}{\ty}}}
      {\typed{\tyctx; \exprctx}{\tapp{e}}{\subst{\ty}{\ty'}{\alpha}}}
      \and
    \end{mathpar}
  \end{block}
\end{frame}

\begin{frame}{Logical relations}
  \begin{block}{Motivation}
    \begin{itemize}
      \item type safety: first step towards more complex properties
      \item logical relations scale well
            \begin{itemize}
              \item non-trivial features
              \item language properties
            \end{itemize}
      \item well-suited for proof assistant
    \end{itemize}
  \end{block}
  \begin{block}{Definition (excerpt)}
    Interpretation context: \(\sctx ::= \empctx\ |\ (\alpha \mapsto P) :: \sctx\), \\
    where $P \in (\mathrm{Expr} \rightarrow \PP)$.
    \begin{flalign*}
      \lrv{\sctx}{\alpha} \eqdef&\ \sctx(\alpha)(v)\\
      %
      \lrv{\sctx}{\tyforall{\alpha}{\ty}}
      \eqdef&\
              \exists e.~v = \tlam{e} \wedge
              ( \forall P.~\safe_{\lrp{((\alpha \mapsto P)::\sctx)}{\ty}}(e))
  \end{flalign*}
  \end{block}
\end{frame}

\begin{frame}{Type safety}
  \begin{block}{Type safety with logical relation}
    \begin{enumerate}
      \item any well-typed closed term is in the logical relation
            \[\forall e,\ \ty.\ \typed{\empctx}{e}{\ty} \Rightarrow \lr{\empctx}{e}{\ty} \]
      \item  any term in the logical relation is safe
            \[\forall e,\ \ty.\ \lr{\empctx}{e}{\ty} \Rightarrow \safe(e) \]
   \end{enumerate}

  \end{block}

  \begin{block}{Fundamental Theorem of the Logical Relation}
    For any \((\typed{\tyctx;\exprctx}{e}{\ty})\),
    \[
      \forall \sctx,\ \sfun.\ (\sfun \Mapsto_{P} \exprctx) \Rightarrow \lr{\sctx}{\ty}{\sfun(e)}
    \]
    with $P = \lambda \ty,~e.~ \lr{\sctx}{\ty}{e}$.
  \end{block}
\end{frame}

% In this part, we will talk a bit more about the difference between a paper proof
% and an actual implementation. In particular, we will see what are the challenges
% and why the Coq implementation is not just straightforwardly copy-paste the
% paper proof
\section{Implementation}%

% THE ISSUE WITH BINDERS

% What is wrong with the named binders

\begin{frame}
  \frametitle{De Bruijn binders}
  \begin{block}{Named binder}
    \begin{itemize}
      % we recall that the name of the bound variables does not matter
      \item Equality up-to alpha renaming
      % it has to be capture-avoiding, ie. the substitution does not capture free variable
      % possibly rename the variables
      \item Capture-avoiding substitution
      % parallel substitution is actually tedious to manipulate in the proofs
      \item Good for presentation, bad for implementation
    \end{itemize}
  \end{block}

  % binders in implementation is actually a well-known problem
  % one solution is De Bruijn technique
  \begin{block}{De Bruijn indices}
    \begin{itemize}
      \item Nameless, unique, canonical representation
      \item Variable points directly to its binder
      \item $k$ is bound to the $k$-th enclosing binder
    \end{itemize}
  \end{block}

  % example to illustrate how to read a term with nameless binder
  % \begin{block}{Example}
    % \begin{itemize}
    %   \item \(\lam{x}{\pair{x}{y}}
    %         \equiv \lam{}{\pair{0}{1}} \) if $1 \mapsto y$ in the context
    %   \item \(\lam{x}{\pair{x}{\lam{y}{\pair{x}{y}}}}
    %         \equiv \lam{}{\pair{0}{\lam{}{\pair{1}{0}}}} \)
    % \end{itemize}
    % \end{block}

  \begin{block}{Example}
    \begin{center}
      \LARGE{
        \only<1>{\(\lam{f}{\app{(\app{f}{y})}{(\lam{x}{\app{f}{x}}})} \)}
        \only<2>{\(\lam{}{\app{(\app{0}{y})}{(\lam{x}{\app{1}{x}}})} \)}
        \only<3>{\(\lam{}{\app{(\app{0}{y})}{(\lam{}{\app{1}{0}}})} \)}
        \only<4>{\(\lam{}{\app{(\app{0}{1})}{(\lam{}{\app{1}{0}}})} \)}
      }
    \end{center}
  \end{block}

  % example to illustrate how to substitute with binders
  % \begin{block}{Substitution}
  %   \begin{itemize}
  %   \item $\subst{\pair{0}{1}}{\tt}{} = \pair{\tt}{0}$
  %   \item $\subst{(\lam{}{\pair{0}{1}})}{\tt}{} = (\lam{}{\pair{0}{\tt}})$
  %   \end{itemize}
  % \end{block}
\end{frame}

\begin{frame}
  \frametitle{Modifications}
  \begin{block}{Semantic}
    \begin{itemize}
      \item Expression and type variable De Bruijn indices
            $k \in \NN$, $\alpha \in \NN$
      \item Expression variable context ordered sequence
            $\exprctx ::= \empctx\ |\ \ty,\exprctx$
    \end{itemize}
    \begin{mathpar}
      \fbox{\typed{\exprctx}{e}{\ty}} \\
      \inferH
      {T-TAbs-DeBruijn}
      {\typed{(\mathrm{map}~(+1)~\exprctx)}{e}{\ty}}
      {\typed{\exprctx}{\tlam{e}}{\tyforall{}{\ty}}}
    \end{mathpar}
  \end{block}

  \begin{block}{Logical relation}
    \begin{itemize}
      \item Interpretation context ordered sequence
            $\sctx ::= \empctx\ |\ P::\sctx$
    \end{itemize}
    \vspace{-0.8em}
    \begin{flalign*}
      \lrv{\sctx}{\tyarrow{\ty_{1}}{\ty_{2}}}
      \eqdef&\ \ldots\wedge
              ( \forall v'.~\lr{\sctx}{\ty_{1}}{v'} \Rightarrow
              \safe_{\lrp{\sctx}{\ty_{2}}}(\subst{e}{v'}{}))\\
              %
      \lrv{\sctx}{\tyforall{}{\ty}}
      \eqdef&\ \ldots\wedge
              ( \forall P.\ \safe_{\lrp{(P::\sctx)}{\ty}}(e))
    \end{flalign*}
  \end{block}
\end{frame}

% THE SUBSTITUTION LEMMA

\begin{frame}
  \frametitle{(Generalized) Substitution lemma}
  \begin{block}{Original Lemma}
    For any $\sctx$, $\ty$, $\ty'$ and $v$,
    \[
      \lrv{\sctx}{\subst{\ty}{\ty'}{}}
      \Leftrightarrow
      \lrv{(\lrp{\sctx}{\ty'}::\sctx)}{\ty}
    \]
  \end{block}

  \begin{block}{Generalized lemma}

    For any $\sctx_{1}$, $\sctx_{2}$, $\ty$, $\ty'$ and $v$,
    \begin{flalign*}
      &\lrv{ \sctx_{1}++\sctx_{2}}{\subst{\ty}{\mathrm{upn}\ (\mathrm{len }\ \sctx_{1})\ \ty'}{}}
      \\ \Leftrightarrow~
      &\lrv{ \sctx_{1}++( \lrp{\sctx_{2}}{\ty'} ::\sctx_{2})}{\ty}
    \end{flalign*}

    \begin{itemize}
      \item the free variable is under \((\mr{len}\ \sctx_{1})\) number of \(\forall.\)
      \item proof method
            \begin{itemize}
              \item polymorphic case straightforward
              \item type variable case harder
            \end{itemize}
    \end{itemize}
  \end{block}
\end{frame}

% \section{Extensions}%

\begin{frame}
  \frametitle{Different approach}
  \begin{block}{Iris}
    \begin{itemize}
      \item framework for higher-order separation logic
      \item step indexed
      \item logic of resources
    \end{itemize}
  \end{block}

  \begin{block}{Extensions of System F}
    \begin{itemize}
      \item recursive type (guarded by \emph{later modality})
      \item mutable states (managed with resource describing the heap)
    \end{itemize}
  \end{block}
\end{frame}


\begin{frame}
  \frametitle{Conclusion}
  \begin{itemize}
    \item type safety of System F using logical relation
    \item implementation in Coq
          \begin{itemize}
            \item binder representation
            \item substitution lemma
            \item implementation choice (semantic)
            \item free theorems
          \end{itemize}
    \item extensions
          \begin{itemize}
            \item implement in the Iris
            \item extend System F
            \item logical relation for other language properties
          \end{itemize}
  \end{itemize}
\end{frame}

\appendix
\begin{frame}{SystemF - Syntax and CBV semantic}
  $x,y \in \mathrm{string}$

  $e ::= x\
  |\ \tt\
  |\ \pair{e}{e}\
  |\ \fst{e}\
  |\ \snd{e}\
  |\ \lam{x}{e}\
  |\ \app{e}{e}\
  |\ \tlam{e}\
  |\ \tapp{e}\
  $

  $v ::= x\ |\ \tt\ |\ \lam{x}{e}\ |\ \tlam{e}$

  $\ctx ::= \empctx\
  |\ \fst{\ctx}\
  |\ \snd{\ctx}\
  |\ \pair{\ctx}{e}\
  |\ \pair{v}{\ctx}\
  |\ \app{\ctx}{e}\
  |\ \app{v}{\ctx}\
  |\ \tapp{\ctx}\
  $

  \begin{mathpar}
    \inferH
    {E-Fst}
    {  }
    {\fst{\pair{v_{1}}{v_{2}}} \hstep v_{1}}
    \and

    \inferH
    {E-Snd}
    {  }
    {\snd{\pair{v_{1}}{v_{2}}} \hstep v_{2}}
    \and

    \inferH
    {E-App}
    {  }
    { \app{(\lam{x}{e})}{v} \hstep \subst{e}{v}{x}}
    \and

    \inferH
    {E-TApp}
    {  }
    { \tapp{\tlam{e}} \hstep e}
    \and

    \inferH
    {E-Step}
    {e \hstep e'}
    {\ctxh{e} \step \ctxh{e'}}
  \end{mathpar}
\end{frame}

\begin{frame}{SystemF - Typing judgement}
 \begin{footnotesize}
  $\alpha, \beta \in \mathrm{string}$

  $\ty ::= \tyunit\
  |\ \alpha\
  |\ \typair{\ty}{\ty}\
  |\ \tyarrow{\ty}{\ty}\
  |\ \tyforall{\alpha}{\ty}\
  $

  $\exprctx ::= \empctx\ |\ x:\ty,\exprctx$

  $\tyctx ::= \empctx\ |\ \alpha,\tyctx$

  $\fbox{\typed{\tyctx; \exprctx}{e}{\ty}}$
  \begin{mathpar}
    \inferH
    {T-Unit}
    {  }
    {\typed{\tyctx; \exprctx}{\tt}{\tyunit}}
    \and

    \inferH
    {T-Var}
    {\exprctx(x) = \ty}
    {\typed{\tyctx; \exprctx}{x}{\ty}}
    \and

    \inferH
    {T-Prod}
    {\typed{\tyctx; \exprctx}{e_{1}}{\ty_{1}}
      \\
      \typed{\tyctx; \exprctx}{e_{2}}{\ty_{2}}}
    {\typed{\tyctx; \exprctx}{\pair{e_{1}}{e_{2}}}{\typair{\ty_{1}}{\ty_{2}}}}
    \and

    \inferH
    {T-Fst}
    {\typed{\tyctx; \exprctx}{e}{\typair{\ty_{1}}{\ty_{2}}}}
    {\typed{\tyctx; \exprctx}{\fst{e}}{\ty_{1}}}
    \and

    \inferH
    {T-Snd}
    {\typed{\tyctx; \exprctx}{e}{\typair{\ty_{1}}{\ty_{2}}}}
    {\typed{\tyctx; \exprctx}{\snd{e}}{\ty_{2}}}
    \and

    \inferH
    {T-Abs}
    {\typed{ \tyctx; x:\ty_{1}, \exprctx}{e}{\ty_{2}}}
    {\typed{ \tyctx; \exprctx}{\lam{x}{e}}{\tyarrow{\ty_{1}}{\ty_{2}}}}
    \and

    \inferH
    {T-App}
    {\typed{\tyctx; \exprctx}{e_{1}}{\tyarrow{\ty_{2}}{\ty_{1}}}
      \\
      \typed{\tyctx; \exprctx }{e_{2}}{\ty_{2}}}
    {\typed{\tyctx; \exprctx }{\app{e_{1}}{ e_{2}}}{\ty_{1}}}
    \and

    \inferH
    {T-TAbs}
    {\typed{\alpha, \tyctx; \exprctx }{e}{\ty}
      \\
      \alpha \mathrm{\ is\ not\ free\ in\ }\exprctx}
    {\typed{\tyctx; \exprctx}{\tlam{e}}{\tyforall{\alpha}{\ty}}}
    \and

    \inferH
    {T-TApp}
    {\typed{\tyctx; \exprctx}{e}{\tyforall{\alpha}{\ty}}}
    {\typed{\tyctx; \exprctx}{\tapp{e}}{\subst{\ty}{\ty'}{\alpha}}}
    \and
  \end{mathpar}
 \end{footnotesize}
\end{frame}

\begin{frame}
  \frametitle{Substitution lemma}
  \begin{block}{Lemma}
    For any $\sctx$, $\ty$, $\ty'$ and $v$,
    \[
      \lrv{\sctx}{\subst{\ty}{\ty'}{}}
      \Leftrightarrow
      \lrv{(\lrp{\sctx}{\ty'}::\sctx)}{\ty}
    \]
  \end{block}

  \begin{block}{Polymorphic case}
    \only<1-2>{$\forall \sctx,\ \ty,\ \ty',\ v$.}
    \only<3->{$\forall \sctx,\ \ty,\ \ty',\ v,\ P$.}
    \vspace{-1em}
    \only<1>{
      \[
        \lrv{\sctx}{\subst{(\tyforall{}{\ty})}{\ty'}{}}
        \Leftrightarrow
        \lrv{(\lrp{\sctx}{\ty'}::\sctx)}{\tyforall{}{\ty}}
      \]
    }

    \only<2>{
    \vspace{-0.2em}
      \[
        \lrv{\sctx}{(\tyforall{}\subst{\ty}{\mr{up}~\ty'}{})}
        \Leftrightarrow
        \lrv{(\lrp{\sctx}{\ty'}::\sctx)}{\tyforall{}{\ty}}
      \]
    }

    % There was a typo in the report !!
    \only<3->{
    \vspace{-0.2em}
      \[
        \lrv{P::\sctx}{\subst{\ty}{\mr{up}~\ty'}{}}
        \Leftrightarrow
        \lrv{P::(\lrp{\sctx}{\ty'}::\sctx)}{\ty}
      \]
    }
  \end{block}

 \begin{center}
\only<4>{
   \LARGE{Too weak }
}
 \end{center}
\end{frame}


\begin{frame}
  \frametitle{Generalized substitution lemma}
  \begin{block}{Generalized lemma}

    For any $\sctx_{1}$, $\sctx_{2}$, $\ty$, $\ty'$ and $v$,
    \begin{flalign*}
      &\lrv{ \sctx_{1}++\sctx_{2}}{\subst{\ty}{\mathrm{upn}\ (\mathrm{len }\ \sctx_{1})\ \ty'}{}}
      \\ \Leftrightarrow~
      &\lrv{ \sctx_{1}++( \lrp{\sctx_{2}}{\ty'} ::\sctx_{2})}{\ty}
    \end{flalign*}
    where $\subst{\ty}{\mathrm{upn}\ \kappa\ \ty'}{}$ modify the substitution such
    that it has already gone through $\kappa$ binders.
  \end{block}

  \begin{block}{Proof modification}
    Inductive case: type variable $\ty = k$
   \begin{itemize}
     \item $k < (\mathrm{len }\ \sctx_{1})$ --- no renaming
     \item $k = (\mathrm{len }\ \sctx_{1})$ --- exactly the substitution
     \item $k > (\mathrm{len }\ \sctx_{1})$ --- renaming
   \end{itemize}
  \end{block}

  % Instantiate the lemma with $\sctx_{1} = \empctx$
\end{frame}

\begin{frame}
  \frametitle{Properties}
  \begin{block}{Normalization}
    A term $e$ reduces to the value $v$:
    \vspace{-0.8em}
    \[e \Downarrow v \eqdef e \mstep v\]

    A term $e$ normalizes if it reduces to a value:
    \vspace{-0.8em}
    \[\norm(e) \eqdef \exists v \in \val.~e \Downarrow v\]
  Proved for STLC using logical relation in my previous talk.
  \end{block}

  \begin{block}{Contextual equivalence}
    Two expressions are contextually equivalent if no context can differentiate them.
    \vspace{-0.8em}
    \begin{flalign*}
      \typed{\tyctx';\exprctx'}{&e_{1} \approx^{\mathrm{ctx}} e_{2} }{\ty'} \eqdef \\
                                &\forall \ctx\ :\ (\tyctx;\exprctx \vdash \ty) \rightarrow (\empctx;\empctx \vdash \tyunit).\\
                                &(\ctxh{e_{1}} \Downarrow v \Leftrightarrow \ctxh{e_{2}} \Downarrow v)
    \end{flalign*}
  \end{block}
\end{frame}

\end{document}
